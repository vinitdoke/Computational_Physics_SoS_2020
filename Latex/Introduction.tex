\section{Introduction}
	\par With greater strides being made in Experimental Physics, we are witnessing data volumes that were never seen before. For instance, the \href{https://home.cern/news/news/computing/cern-data-centre-passes-200-petabyte-milestone}{Large Hadron Collider generates petabytes of data per second.} This warrants the use of computers to analyse, reformat and simulate various tasks in Physics. This is the field of study represented by Computational Sciences, Computational Physics being the application in Physics.\medskip
	\par Most numerical calculations in physics fall into one of several general categories, based on the mathematical operations that their solution requires. Examples of common operations include calculation of integrals and derivatives, linear algebra tasks such as matrix inversion or the calculation of eigenvalues, and the solution of differential equations, including both ordinary and partial differential equations. 
     If we know how to perform each of the basic operation types then we can solve most problems we are likely to encounter as physicists. I shall study individually the computational techniques used to perform various operations and then apply the solution to wide range of physics problems.
\subsection{Using Python}
Being an interpreted language, Python provides a much more natural coding experience. The plethora of support libraries available for it make it an ideal choice for scientific computing purposes. Some of the libraries required for this project include \textbf{Numpy, Pandas, Matplotlib, VPython, SciPy} etc. 
	\par Though many of these libraries provide in-built functionalities for carrying out various numerical operations, we shall follow a ground up approach towards these operations, exploring the methods followed to port them to a computer.
\subsection{Numerical Methods}
We shall see various methods of numerical evaluations of maths operations and then deal with the ways of calculating the errors produced due to either approximations or precision value errors (inherent to Python due to way the data types are stored). I shall be simultaneously solving problems pertaining to varied areas of Physics related to the mathematical methods which are linked here as follows:
 
\begin{enumerate}
	\item \hyperlink{section.2}{Data Visualisation}
	\item \hyperlink{section.3}{Integration}
	\item \hyperlink{section.4}{Solutions Of Non-Linear Equations}
	\item \hyperlink{section.5}{Ordinary Differential Equations}
	
	\item \hyperlink{section.6}{Random Processes}
	\item \hyperlink{section.6}{Monte-Carlo Methods}
	
	%\item Linear Equations
	%\item Non-Linear Equations
\end{enumerate}