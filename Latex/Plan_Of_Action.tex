\documentclass[11pt,a4paper,leqno]{article}
\usepackage[width=15.00cm, height=27.00cm]{geometry}
\usepackage{authblk}
%opening
\title{\textbf{\huge Computational Physics}}

\author[1]{ Vinit P. Doke}
\affil[1]{{
		Department of Physics\\ 
		Indian Institute of Technology\\
		Powai, Mumbai 400076 \\
		{Email:} \texttt{vinitdoke@gmail.com , 190260018@iitb.ac.in
		}.}
}
\date{{ Mentor: Chaitanya Kumar }}



\begin{document}

\maketitle
\section{{\textbf{{\LARGE         Plan Of Action}}}}

	
\subsection{Objective}
 {\large The objective of this endeavour is to get an idea of the roll of Computation in solving Physics problems. Computational Physics complements the areas of theory and experimentation in traditional scientific investigation. Final goal would be to apply the techniques to predict outcome by simulation of the physical phenomenon.
 	
\subsection{Resources}
{\large 
\begin{itemize}
	\item \textbf{Computational Physics: Problem Solving with Computers} [RUBIN H. LANDAU]
	\item \textbf{Computational Physics} [MARK NEWMAN]
	\item \textbf{www.w3schools.com} [for learning Python]
\end{itemize}}
\subsection{General Path to be followed :}
\begin{enumerate}
	\item Learning coding in Python.
	\item Learning the necessary libraries in Python:
	\subitem NumPy
	\subitem SciPy
	\subitem Pandas
	\subitem Matplotlib
	\item Implementing Math through code:
	\subitem Integrals and Derivatives
	\subitem Solving Linear and Non-Linear Equations
	\subitem ODEs and PDEs
	\subitem etc.
	\item Monte-Carlo Simulations
	\item Application of the techniques learned so far
\end{enumerate}





\end{document}
