


	
\subsection{Objective}
 {\large The objective of this endeavour is to get an idea of the roll of Computation in solving Physics problems. Computational Physics complements the areas of theory and experimentation in traditional scientific investigation. Final goal would be to apply the techniques to predict outcome by simulation of the physical phenomenon.
 	
\subsection{Resources}
{\large 
\begin{itemize}
	\item \textbf{Computational Physics} [MARK NEWMAN]
	\item \textbf{Computational Physics: Problem Solving with Computers} [RUBIN H. LANDAU]	
	%\item \textbf{www.w3schools.com} [for learning Python]
\end{itemize}}
\subsection{Completed Topics :}
\begin{enumerate}
	\item Learning coding in Python.                  
	\item Learning the necessary libraries in Python:
	\subitem NumPy
	\subitem SciPy
	\subitem Pandas
	\subitem Matplotlib
	\item Implementing Math through code:
	\subitem Integrals and Derivatives
\end{enumerate}
\subsection{Timeline for Post-Mid-Term:}
\begin{center}
\begin{tabular}{|c|c|}
	\hline
	Linear and Non-Linear Equations &  6-05-2020 \\
	\hline
	Fourier Transform & 10-05-2020 \\
	\hline
	ODEs and PDEs & 15-05-2020 \\
	\hline
	Monte-Carlo Simulation & 19-05-2020 \\
	\hline
	Application of the Techniques learned so far & 23-05-2020 \\
	\hline
	Buffer and Final Report  & 28-05-2020\\
	\hline
\end{tabular}
\end{center}




